% $Id: AllegProposal.tex,v 1.8 2000/07/05 21:02:12 culver Exp $
% AllegProposal.tex
% by A. Thall
% 13. Feb 2003
%
% Small edits and a few additions made by R. Roos
% 21 Jan 2007
% Most particularly, the "box" around the thesis statement has been removed,
% section titles have been modified. The section named "Prior work II" has
% been commented out. The \topmargin has been changed to -.5in and the
% change to \parindent has been commented out.
% The filename "nausicaa.eps" has been changed to simply "nausicaa" so that
% pdflatex can be used on the file (and a file named "nausicaa.pdf" has
% been created using the "epstopdf" command).
% Several subsections have been added to illustrate subsection usage.
% The word "comp" has been replaced by "project" or "thesis" throughout.
% Other small changes have been made.
%
% This document provides a sample Senior Project Proposal template for use
% by students in Allegheny's CS and Applied Computing programs.

\NeedsTeXFormat{LaTeX2e}
\documentclass[11pt]{article}

%The following is used by WinEdt to set up cross-referencing to the BibTeX files
%It is NOT commented out---the comment lets it be simply ignored by non-WinEdt LaTeX compilers

%GATHER{mybibtexDB.bib}

\usepackage{setspace}
\usepackage{amsmath}
\usepackage{amssymb}
\usepackage{epsfig}
\usepackage{fancybox}
\usepackage{listings}
\usepackage{algo}
\usepackage{url}

\setlength{\textheight}{9in}
\setlength{\textwidth}{6in}
\setlength{\oddsidemargin}{.25in}
\setlength{\topmargin}{-.5in}  % changed from -.25 by RSR on 1/21/07
%\parindent .5in    % commented out by RSR 1/21/07

%put words in the hyphenation statement if you want to enforce
%how LaTeX should break them (or not) at the end of a line.
%\hyphenation{repre-sen-tations problems exact linear}
\hyphenation{itself}

%%%%%
%% Commented out -- RSR, 1/21/07
%%%%%
% The following provides a box to surround the thesis statement
%\newenvironment{Thesis}%
%{\begin{Sbox}\begin{minipage}{.95\linewidth}}%
%{\end{minipage}\end{Sbox}\begin{center}\fbox{\TheSbox}\end{center}}

\title{A Senior Thesis Proposal Template for CS Students at Allegheny}
\author{Stu Dent \\ Possible thesis advisor:  I.\ M.\ Firstreader}

\begin{document}

% You can specify a language and other options for
% the code-formatting "listings" package
\lstset{language=C++,basicstyle=\small,
        stringstyle=\ttfamily,showstringspaces=false}

\singlespace
\maketitle

\begin{abstract}                % ~350 words max
A \LaTeX\ template for Senior Project proposals would be of benefit to
students who find the organizational aspects of such a project
difficult or unclear.  This paper proposes itself as just such a
template.  Using this template as a reference guide, students should be
able to understand the form and function of the different parts of a
formal thesis proposal; they may, of course, choose to deviate from it
as necessary or desirable in the context of their own project
proposals.  Future work will provide another template for the formal written
thesis itself, complete with a {\tt gatorthesis.sty} style-file.
\end{abstract}

\doublespace
% This sets section-numbering to only include Section and Subsection numbers
\setcounter{secnumdepth}{2}

\section{Introduction}\label{ch:overview}

Creation of a formal research proposal is a daunting task; while authors
such as Zobel~\cite{zobel:97} discuss the basic mechanics of technical
writing, and writers such as Griffith et al.~\cite{griffiths:97}
explain the basics of using \LaTeX\ for such writing, there is little
guidance on the actual structure of a research proposal.
This structure may vary considerably depending on the nature of the
research topic---whether it consists of a coding project, a theoretical
study, a literature survey or other acceptable topics.  Despite their
differences, all proposals share common features, and this paper provides
a \LaTeX\ template suitable for most such projects.  It is expected that
a user will modify this template according to the specifics of the
proposed research; in particular, boiler-plate section headings
and subheadings should be replaced by informative ones specific to
the topic.

\subsection{Components of a Proposal}
The introduction of a proposal typically acquaints the reader with
the nature of problem being addressed and the basic idea of the
project being proposed to address it.  It should hit many of the points
of the abstract while giving more of the motivation and need for the
work being proposed.  It should then lay out the structure of the
paper, telling the reader what will be found in the sections which
follow.  Thus, the next sections should address the following topics:
\begin{itemize}
\item
the thesis statement and research objectives;
\item
prior work and current art---other approaches to the problem and
their drawbacks;
\item
methodology and experiment---including background research
to be done, necessary equipment and algorithmic techniques, and general
plan-of-attack;
\item
analysis and validation---how results will be judged and interpreted;
\item
a time-table for achieving key objectives, and one or more fallback plans
in cases where goals are not met by specific target dates;
\item concluding remarks.
\end{itemize}
The proposal should also include a bibiography/reading-list.  In
this case, it is permissible to include references which are not directly
cited in the text.  (You can use the \verb+\nocite{*}+ command before the
bibliography section, as below, to include all Bib\TeX\ references
from a database file in the bibliography.)  You should still cite
references in the text as appropriate.

\section{Prior Work}

Depending on the nature of the topic, prior work should either precede
or follow the thesis statement.  Some topics will require the
background information to put the thesis proposal in context.  Others
are best served by giving the thesis statement first and then contrasting
and comparing it with prior work in the field.

\subsection{\TeX}
If this were a real proposal dealing with how to write proposals,
it might be appropriate to say a few words
about Knuth, inventor of the \TeX\ typesetting language \cite{knuth:84}.
Description of Knuth's contribution goes here.

\subsection{\LaTeX\ (Lamport, 1984)}
Description of Lamport's contribution \cite{lamport:94} goes here.

\subsection{Other Work}
Description of other contributions leading up to this thesis goes here,
e.g., Zobel's book on writing for computer science \cite{zobel:97}.

\section{Thesis}

It is difficult for students to begin a proposal without an adequate guide;
my project, therefore, will demonstrate the following thesis:
%\begin{Thesis}
\singlespace
\begin{quote}
It is both simple and useful to provide students with a \LaTeX\ template
for their formal thesis proposals.  This will result in far fewer
questions from students uncertain as to what belongs in their proposals.
Such a template will be of benefit to students in Computer
Science/Applied Computing as well as to any others willing and able
to use \LaTeX\ for their work.
\end{quote}
%\end{Thesis}
\doublespace

A formal thesis statement should be a \emph{falsifiable} statement about
the goal you will attempt to achieve with your research project.
For a purely scientific project, this is the hypothesis you are testing
with your research.  For an applied programming project, it is usually a
statement about the feasibility and correctness of your approach and
the advantages it has over other approaches.  For a survey or study,
it is usually a statement regarding the need or usefulness of such
a study, its intended audience, and so on.

Often, you may want to include an itemized list of goals you plan to
achieve.  Thus, this paper has the goals of
\begin{itemize}
\item
helping students, by ridding their minds of confusion;
\item
helping faculty, too, by eliminating their need to say ``No, no, no!
You need to include a section on \emph{bleh} and \emph{blah}!
I told \emph{everybody} that.  I thought I told everybody that.  Arrgghh!!!''
\end{itemize}
In summary, this section gives the main idea of what you propose,
with the goals and contributions it will make to the field; the
details of proposed implementation and methodology will be given later
in Sec.~\ref{sec:implem}.

%\section{Prior work II:  advantages of the proposed work over other approaches}

If you didn't already discuss prior work, this is a good place to do so.
%You might include these remarks, if brief, in the preceding section.

\section{Implementation and Methodology}\label{sec:implem}

Here, you should lay out the details of how you propose to solve the
problem and otherwise conduct the research necessary to support your
thesis.  Include details regarding hardware and software you will use,
resources you will draw on, algorithms you will implement, and other
ideas about how you will accomplish your task.  It is inevitable that
your final work will deviate from earlier plans, as you research your
topic, learn new methods, and discover what works as expected.

\subsection{Using Tables and Figures}
Use tables and figures as appropriate; a picture can explain a lot
in very compact form, and can keep readers interested.  Avoid things
that are merely flash and do not
\begin{figure}
\centering
\begin{tabular}{l l l}
\epsfxsize=1.8in\epsffile{Nausicaa} &
\epsfxsize=1.8in\epsffile{Nausicaa} &
\epsfxsize=1.8in\epsffile{Nausicaa} \\
\end{tabular}
\caption{Include images in a proposal as appropriate}\label{fig:nausicaa}
\end{figure}
add any relevant information.  (See Fig.~\ref{fig:nausicaa}.)

\subsection{Working with Code and Pseudocode}
If you have source code to include, you can do so using the
\emph{listings} package,
which will format short inline code fragments such as

\singlespace
\begin{lstlisting} {}
for (int i = 0; i < n; i++)
    cout << "It's easy to add source code to LaTeX documents";
\end{lstlisting}
or simply using the \emph{verbatim} environment, which gives a Courier font to literal text:
\begin{verbatim}
for (int i = 0; i < n; i++)
    cout << "It's easy to add source code to LaTeX documents";
\end{verbatim}
\doublespace
The listing environment is good for longer code examples and for use in figures,
such as in Fig.~\ref{fig:surprisecode}.

\begin{figure}[t]
\lstset{basicstyle=\scriptsize}
\lstinputlisting{surprise.c}
\caption{This mystery code (\copyright 1987 Roemer B.\ Lievaart) was
included from a source file.  The \LaTeX\ file also shows how to
change the font-size for a code-listing.}\label{fig:surprisecode}
\end{figure}

Another frequent need is to include algorithms written in pseudocode.
The \emph{algo} package can be used to format algorithms presentably in your
documents; Fig.~\ref{fig:mutation_adequacy} on
Pg.~\pageref{fig:mutation_adequacy} shows an example of this.
For more options on these packages, consult online resources.

\begin{figure}[t]
\begin{algorithm}{CalculateMutationAdequacy}[T, P, M_o]
{
\qcomment{Calculation of Strong Mutation Adequacy}
\qinput{Test Suite $T$; \newline
        Program Under Test $P$; \newline
        Set of Mutation Operators; $M_o$
}
\qoutput{Mutation Adequacy Score; $MS(P,T,M_o)$}
}
${\cal D} \qlet {\cal Z}_{n \times s}$ \\
${\cal E} \qlet {\cal Z}_{s}$ \\
\qfor $l \in \mbox{\it ComputeMutationLocations}(P)$ \\
\qdo $\Phi_P \qlet \mbox{\it GenerateMutants}(l,P,M_o)$ \\
\qfor $\phi_r \in \Phi_P$ \\
\qdo \qfor $T_f \in \langle T_1, \ldots, T_e \rangle$ \\
\qdo $R_f^P \qlet \mbox{\it ExecuteTest}(T_f,P)$ \\
$R_f^{\phi_r} \qlet \mbox{\it ExecuteTest}(T_f,\phi_r)$ \\
\qif $R_f^P \neq R_f^{\phi_r}$ \\
\qdo ${\cal D}[f][r] \qlet 1$ \\
\qelse \qif ${\it \mbox{\it IsEquivalentMutant}}(P,\phi_r)$
                                                 \\ \label{equivalent}
\qdo ${\cal E}[r] \qlet 1$ \qfi \qfi \qrof \qrof \qrof \\
$D_{num} \qlet \sum_{r=1}^s pos( \sum_{f=1}^n {\cal D}[f][r] )$ \\ \label{sum1}
$E_{num} \qlet \sum_{r=1}^s {\cal E}[r] $ \\ \label{sum2}
$MS(P,T,M_o) \qlet \frac{D_{num}}{(|\Phi_P| - E_{num})}$ \\ \label{result}
$\qreturn \; MS(P,T,M_o)$
\end{algorithm}
\caption{Algorithm for the Computation of Mutation Adequacy.  This example
pseudocode is by Gregory M.\ Kapfhammer.}
\label{fig:mutation_adequacy}
\end{figure}

\section{Research and Writing Timetable}\label{sec:timetable}

Give a brief overview of how you will proceed to accomplish the project,
including a rough schedule for accomplishing the following tasks:
\begin{itemize}
\item background research,
\item final proposal completion,
\item proposal defense,
\item research intermediate and final goals, and
\item writing of key chapters, leading to final written thesis.
\end{itemize}
The timetable should take into account the actual schedule followed by
the department, CS600 in the fall is devoted to the background research,
final proposal and its defense, beginnings of primary project work and
writing of first two chapters.  CS601 in the spring is spent finishing
the research and the writing and preparing for the thesis defense.

The timetable section may also include contingency plans---see below in
Sec.~\ref{sec:conclusion}.

\section{Conclusion}\label{sec:conclusion}

Concluding remarks may discuss future research directions that will
be made possible when the work succeeds, and possible places where
the work might be cut short (due to difficulties) while still achieving
some of the significant objectives.  The latter might alternatively
be discussed above in Sec.~\ref{sec:timetable}.

The conclusion may also discuss future applications of and extensions
to the thesis work.  After completing this thesis proposal template,
a logical next step is to create a template for the written thesis itself.
Since writing a $50+$ page document requires even more discipline and
organization, this will entail creating a \LaTeX\ style file, tentatively
named \verb+gatorthesis.sty+, which will be called by a
\verb+\usepackage{gatorthesis}+ command.  This will provide automatic
formatting for title pages, abstract and acknowledgement pages, Tables
of Contents and Figures, chapter headings, and so forth.  For
an experienced \LaTeX\ user, with a Bib\TeX\ database already assembled
for the proposal, this will make thesis writing go a lot smoother.

This paper has shown only one of many different ways you might structure
your project proposal.  Individual proposals will vary depending on the
nature of the project, but most of them will have the same essential
components.  You will have many other occasions to write
proposals---whether for graduate projects and theses, grant proposals,
or industry-related projects.  The important thing is to express
your ideas in a concise and effective fashion, so that a reader is
neither confused nor bored nor irritated.  \emph{Too long and wordy}
is as bad as \emph{too short and lacking in detail}; grammatical
and spelling errors are likewise unacceptable.  Write fast, rewrite
thoroughly, and proofread religiously.

\pagebreak

% This includes all references from the BibTeX file in the bibliography
\nocite{*}

\begin{spacing}{1}
  \bibliographystyle{plain}
  \bibliography{mybibtexDB}
\end{spacing}

\end{document}
