% Methodology.tex
%==============================================================================
\section{Methodology}
\label{sec:methodology}
%==============================================================================

%-- Context
\noindent 

The project will be developed in the laboratory of the Research Group of Applied Computing (GCA), in Department of Exact Sciences and Engineering, Uniju\'{i} University.
The project will be divided into cycles, within which a set of activities must be performed. At the beginning of each cycle, a planning meeting will be held, in which supervisor together with the proponent will define and prioritize the points that will be worked during the cycle, select the activities that must be implemented and the products or artefacts that will result from the work done during that cycle. cycle.

At the end of each cycle, the proponent presents the progress of his / her work, through a meeting or seminar, depending on the relevance of the work done for the research progress. In such meetings, besides being a supervisor and a coordinator, GCA teachers and students will be able to participate, in person and through video conference. The objective is to promote discussions, disseminate knowledge, identify possible inconsistencies and collect new ideas about the work carried out in the cycle that ends.

In addition, at the end of the cycle, the proponent and her supervisor will hold an evaluation meeting to assess the timing, productivity, learning and planning of the next cycle.
Then a new one begins, repeating the process until the whole schedule presented in Section~\ref{sec:schedule} is fulfilled.
It is worth mentioning that one of the artefacts that will be generated at the end of the cycle will be the research report, which will have periodicity defined by the supervisor.

%==============================================================================
\subsection{Infrastructure}
\label{subsec:infrastructure}
%==============================================================================
